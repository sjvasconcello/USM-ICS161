\newpage

\section{Optativos Así Nomás}

\subsection{Resumen Video: Entra en vigencia la Ley de Pago a 30 Días}
Problemas de las pequeñas empresas, la burocracia para realizar el pago. Generaba perdidas de eficiencia por el entorpecimiento del pago. Además, genera problemas de liquides dentro de la empresa.
\subsection{Larraín: “Decir que creceremos en un rango de 3 a 3,5 es una dosis de realismo”}
Dato de vital importancia, antes proyectaron 3,5 y ahora 3, ósea dice que creceremos menos que lo poco que se esperaba. Fin del Dato de Vital importancia. 
\\
\\
Larraín, en adelante como el sujeto, actual secretario de estado. Comenta que las proyecciones de crecimiento de Chile cambian totalmente por la guerra comercial China-USA. La guerra comercial bajo la compra de cobre por parte de China a Chile, por ende, se factura y recaudan menos billetes para el estado. Aparte de la baja en la cantidad de cobre comprado, tenemos gastos extras en desastres naturales, y en el mundo de la agricultura perdidas por heladas y temporales. 
\subsubsection{Proyecto Tributario}
No saben si el nuevo proyecto tributario frenara el actual crecimiento en la inversión por parte de privados. Que incluye una modernización del actual sistema tributario. 
\subsubsection{Reformas No tributarias (Expectativas de Crecimiento, Promesas de Campaña)}
Proyectos que necesitan aportes monetarios desde las tributaciones de todos los chilenos. También se han retrasado proyectos de reformas temáticas respecto de las condiciones laborales, pensiones, proyectos que incluyen un mayor gasto de recursos por parte del gobierno. 	
Reforma a las pensiones, aumento en la recaudación por parte de los trabajadores para sus cotizaciones obligatorias. Se teme que esta reforma genere un mayor costo en la mano de obra chilena, dado que los sueldos líquidos disminuirán. 
\subsection{Economista jefe del Banco de Chile: “Hoy debemos estar más abiertos que nunca a la posibilidad de que el crecimiento no llegue al 3 este año”}
Se habla de un primer trimestre débil, dado que no se han alcanzado las expectativas de crecimiento necesarias para cumplir la meta proyectada de un 3,5.  Por lo tanto se genera una noticia de que tan bajo fue el trimestre y que expectativas quedan para el resto del año.
\\
Para lograr el 3\% proyectado se debe crecer a eso del 4\% en lo que queda de periodo. Actualmente se crece al 0.4\% (Tasa del primer trimestre)
\\
Esta es una mirada desde el Banco de Chile (Distinto del Banco Central) es decir un poco menos politizada. 
\\
También existen factores internos que explican las bajas tazas de crecimiento. Como lo son el ingreso de mano de obra y por ende una baja en los precios de los trabajos menos calificados. Y el aumento de trabajos semi formales, como lo son Uber, Pedidos Ya… Junto a la automatización de algunos trabajos como el cobro de estacionamiento. 
\subsection{Caen Rendimientos de bonos del tesoro ante mayor nerviosismo por guerra comercial}
Bueno esta noticia era de una línea, casi una afirmación y dice que aumenta la posibilidad de que el BCentral tenga que bajar la tasa de interés (Ya lo hizo ) Dado que la posibilidad de guerra arruina el sistema de resguardar el dinero en una institución de un país. 
\subsection{Wall Street enciende las alarmas por guerra comercial y advierte que una recesión podría venir muy, pero muy pronto. }
Toman opiniones de los principales directores de bancos de USA, ¡podrían tomar de otros países he!, no se qué opinaran en Japón por ejemplo, parte de Asia. Básicamente, China disminuyó su fabricación de TODO, dado a que existe posibilidad que USA, les ponga zarpado impuesto de un 30\% extra al actual y los deje con bocha de productos manufacturados de inventario.
\\
Este duelo a muerte con cuchillos entre USA y China genera ralentización de la economía mundial. Y que los principales analistas de bolsas del mundo, prefieran invertir en bonos del tesoro de países, en vez de empresas. (Esto genera baja de inversión privada y un ciclo sin fin hacia la disminución de crecimiento.)
\\
Toda inversión tiene un riesgo, (finanzas 0) dado el clima político, los riesgos suben, por lo tanto los millonarios no quieren perder su dinero y lo guardan en los colchones de sus casas, en vez de estar creando mas empresas. Dado que el gobierno puede hacer que una empresa quiebre de un momento para otro. 
\subsection{El petróleo se derrumba y las materias primas se ubican en su nivel más bajo del año}
Dato de vital importancia, Commodities son materias primas de bajo procesamiento. Es decir, poco valor agregado. Combustibles, Minerales, Comida, Cositas que puedo sacar de la tierra. Fin del dato. 
\\
\\
Como les dije en la noticia de arriba, cuando los millonarios guardan los billetes en sus casas, que creen que es lo primero que se deja de consumir. Si!! Adivinaron. Cositas de la tierra. Dado que se dejan de hacer cosas de lujo, uso el carbón para quemarlo y no para hacer diamantes entonces, uso menos carbón.  Como el petróleo de los yankisss es considerado como una especie de Oro, dado que existe una cantidad limitada en el mundo. Que su valor baje, afecta las exportaciones de USA, su cantidad de dinero, y su control sobre el mercado. Si son medios tiranos con el combustible estos capitanes América.
\\
\\
Bueno si Asia, disminuyo drásticamente su producción para que los yankisss no los dejaran con el inventario parado. Adivinen aparte de petróleo, que mineral bajara de precio, ya que no se harán tantos cables ?  Correcto ¡ el Cobre, baja en la demanda, exceso de producción el precio, mis notas y el perreo, hasta el piso.  
\subsection{Cobre cae a mínimo de cinco meses ante preocupación por la demanda}

OH! spoiler la noticia anterior. 
\\
\\
China, desclasifico, que tiene una lista de grupos, empresas y extranjeros que consideran poco fiables. Terror para los países tercermundistas dependientes de las dos potencias como lo es Sudamérica. 
\subsection{Tensión global aumenta demanda por activos seguros y el oro llega a máximos de dos meses}

OH! spoiler la noticia anterior, anterior. 
\\
\\
Bueno adivinen bajo el colchón es Oro, bonos del tesoro, Arte, Autos Históricos (Casi arte) y autos de carreras (Arte por Excelencia) aumentan sus precios, dado que aumenta la demanda, lugares seguros para guardar dineros, y unidades limitadas por ende el precio a las nubes.  Bueno, Plata y Platino, también están subiendo de precio, dado que están guardando los billetes en estos metales. 

\subsection{La guerra comercial y sus efectos sobre Latinoamérica}
USA y China se suben los impuestos entre ellos, esto habia bajado el PIB global en varios puntos de porciento e las nuevas medidas que tenia en mente Caeza de pichi generarián algo asi como una baja de 0.7 puntos porcentuales. Esto genera una disminución del crecimiento y una baja de la demannda de recursos basicos a nivel global, atacando principalmente a las economias en crecimiento arcaicas que se dedican a extraer recursos naturales. alias commodities 
	
\subsection{BC sorprende: realiza mayor baja de la TPM en una década y recorta proyección de PIB 2019}
Todos esperaban un crecimiento piola para chile en los proximos años sobre todo pq el 2017 fue un año de bajo crecimiento, entonces si se tiene registro de que un país puede crecer al 4\% si un año crece al 1\% tiene la capacidad de crecer al 4, regresando a su periodo de maxima productividad. 
Hoy proyectan que al 2021, regresaremos al crecimienot del 2018. 
\\
\\
Es como lo mismo de la notica de arriba, ahora todos estan como sorprendidos por la magnitud. Lo explique mas arriba pero en el fono bajar la el interes te permite acceder a una mayor cantidad de dinero hoy. Por ende aumenta la inflacion y tiene algunos otros defectos. 
\\
\\
Luego habla de por el poco cobre que se esta vendiendo, obviamente el precio se va a mantener alto. Dada, la poca cantidad de billetes del tio sam que ingresaran a chilito. En chilito se compran con dolares electricidad, turismo, combustibles, alimentos, vestimentas, casi todo. Por lo tanto la subida del dolar sube el costo de una canasta basica minima por ende, IPC (informacion para comente, es que tambien Chilito exporta varias cosillas. por lo que aumentan los ingresos de los que viven de la fruta por ejemplo. *espacio auspiciado por thomson premium cherries)
\\
A nivel Gob se esperaba una baja de 25 puntos, y no de 50, luego debaten sobre si la magnitud es la correcta versus la alta inflacion. Bueno yo diria que un aumento drasticos en los precios, genera mas efecto sustitucion, que un aumento en en consumo de esos bienes en especificos. 

\subitem
\centering
\\
\textcolor{Red}{Ha y recordad se un caballero como Benito Martínez y no como ese hijo de puta de Bad Bunny}


