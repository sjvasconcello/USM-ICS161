\section{Los diez principios de la economia}

La palabra \textbf{economía} proviene del griego oikonomos, que significa: 'Administración de una casa''. 
La economía existe para hacer gestión de los recursos de la sociedad, esto debido a que estos son escasos. 
La \textbf{escazes} queriere decir que la sociedad tiene recursos limitados, por esto es que la sociedad no puede
producir todos los bienes que decea tener. La {\bf economía} es: \underline{es el estudio de cómo la sociedad gestiona los recursos}.


\begin{description}
  \item [\large 1°- Las personas enfrentan disyuntivas:] Para obtener algo, usualmente se debe renunciar a otra cosa. Terminos importantes son: \underline{Eficiencia}: Significa que la sociedad obtiene lo máximo de sus recursos escasos y \underline{Equidad}: Significa que los beneficios de esos recursos se distribuyen entre los miembros de la sociedad.
  \item [\large 2°- El costo de una cosa es aquello a lo que se renuncia para obtenerla:] Las decisiones requieren comparar costos y beneficios de diferentes alternativas. El \underline{costo de oportunidad} de algo es a lo que renuncias para obtener algo.
  \item [\large 3°- Las personas racionales piensan en terminos marginales:] Las {\bf personas racionales} son individuos que deliberada y sistemáticamente tratan de hacer lo posible para lograr sus objetivos. Los \underline{cambios marginales} son pequeños ajustes que se hacen a un plan de acción existente
  \item [\large 4°- Las personas reponden a incentivos:] Un \underline{incentivo} es aquello que induce a las personas a actuar. Por lo tanto, las personas toman decisiones comparando costos y beneficios, y son los cambios marginales en los costos o beneficios lo que incentiva a las personas a responder. {\bf La decisión de elegir una alternativa sobre otra ocurre cuando los beneficios marginales de esa alternativa superan sus costos marginales}
  \item [\large 5°- El comercio puede mejorar el bienestar de todos:] Las personas se benefician de su capacidad para comerciar entre sí. La competencia se traduce en ganancias para el comercio. El comercio permite que las personas se especialicen en lo que hacen mejor
  \item [\large 6°- Los mercados normalmente son un buen mecanismo para organizar la actividad económica:] Una {\bf economía de mercado} es una economía que asigna recursos a través de las deciciones descentralizadas de muchas empresas y hogares a medida que interactúan en los mercados de bienes y servicios
  \item [\large 7°- El gobierno puede mejorar algunas veces los resultados del mercado:] Los mercados funcionan solo si se resetan los derechos de propiedad. Los \underline{derechos de propiedad} son la capacidad de una persona para poseer y ejercer el control sobre un recurso escaso. Las \underline{fallas del mercado} ocurren cuando el mercado no se puede asignar los recursos de manera eficiente. Cuando el mercado falla, el gobierno puede intervenir para promover la eficiencia y la equidad. La falla de mercado puede ser causada por: Una \underline{externalidad}, que es el impacto de las acciones de una persona o empresa en el bienestar de otra, o por \underline{porder de mercado}, que es la capacidad de un o unos pocos actores de mercado (personas o empresas) para influir indebidamente en los precios del mercado.
  \item [\large 8°- El nivel de vida de un país depende de la capacidad que tenga para producir bienes y servicios:] El nivel de vida se puede medir de diferentes maneras\footnote{Comparando los ingresos persoles o al comparar el valor de mercado total de la producción de una nación}. Casi todas las varoaciones en los niveles de vida se explican por las diferencias en las productividades de los países. La \underline{productividad} es la cantidad de bienes y servicios producidos por cada unidad de trabajo.
  \item [\large 9°- Cuando el gobierno imprime demasiado dinero los precios se incrementan:] La \underline{inflación} es un aumento en el nivel general de precios en la economía. Una de las causas de la inflación es el crecimiento de la cantidad de dinero. Cuando el gobierno crea grandes cantidades de dinero, el valor del dinero cae.
  \item [\large 10°- La sociedad enfrentan a corto plazo una disyuntiva entre inflación y desempleo: ] Esta disyuntiva desempeña un papel clave en el análisis del \underline{ciclo económico}, que son fluctuaciones en la actividad económica, como el empleo y la producción.
\end{description}

\newpage
