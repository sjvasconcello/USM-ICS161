\newpage

\section{Tarea8}

\subsection{UN FUTURO QUE FUNCIONA: AUTOMATIZACIÓN, EMPLEO Y PRODUCTIVIDAD 
}
Todos sabemos que la tecnología es cada vez mas independiente y que existe la posibilidad de integrar muchas tareas y funciones bajo un ente supervisor que puede funcionar las 24 horas sin café, ni errores.  
El costo de este aumento de productividad lo pagaran los trabajos mas simples. Por lo que aumentara la productividad y dejara cerca de 2000 actividades laborales obsoletas. Luego la base de un trabajo ahora sera entregar apoyo a las falencias que pueda tener una maquina. Es decir la primera linea de trabajo sera de alta interacción y control de una maquinaria.  
Principalmente Usa y China tienen la mayor cantidad de puestos de trabajos automatiza-bles, ahora, el costo de ese puesto de trabajo puede ser aun muy inferior al de reemplazo por una maquinaria. sobre todo en las condiciones de China.
\bigskip 
Consideren que la cantidad de humanos en cierta medida es limitada. Por lo que la productividad de un país esta netamente limitada a su capacidad obrera. La automatización es en algunos casos la única opción de aumentar el crecimiento productivo dramáticamente. Dados sus factores productivos. 

\subsection{Automatización en Chile: ¿Cuál es la realidad que hoy vive cada sector?}

3.2 millones de empleos pueden ser substituidos en Chile, bueno, que esperaban ? nos dedicamos a sacar piedras del suelo y frutas de los arboles. 
El Sector minero es el principal sector en ser atacado, dado que ya existe una automatización semi obligatoria por la seguridad.
\smallskip 
El sector financiero no se queda atrás, es decir aun no existe una bolsa inteligente en Chile, como ya existe la opción en bolsas de los mercados de los grandes países. En USA se dedican a crear algoritmos de compra y venta y no a realizan las tareas de compra directamente. 
En la agricultura igual, los cambios climáticos exigen implementación de tecnologías para asegurar una cosecha. 
En el mundo de los lugares con exhibición y ventas, se prevé que no sera tanta la perdida de empleo ya que la modernización, genero la posibilidad de obtener cualquier producto a domicilio, mas el e-comerce, llevaría al fin de estos servicios de forma definitiva. (Dato de poca importancia Amazon invento la modalidad de que te envían a tu  casa productos que quizás quieras... y si no lo quieres lo regresas. Si lo quieres te lo cobran).0

\subsection{The great jobs boom} 
Parte diciendo como que el empleo sobre calificado, así como los puestos de trabajos mejores pagados han sufrido un aumento sin precedentes. El espíritu de la data, cambio las necesidades y aumento los puestos de trabajo. Muestra que existen muchos precedentes sobre que el trabajo va en aumento en Europa y USA. El cambio generacional, post un recesión la gente que va a trabajar llega diferente, se califica, busca alternativas para encontrar empleo.  Es decir si ya se dejo de necesitar mano de obra de poca inducción para recolectar datos, hoy en día se pude colocar un sensor, esto ha generado boche de necesidad de analistas por ejemplo. 
Globalmente se cree que el mundo del trabajo tendrá una re formulación completa, pasando de una jornada diaria en una empresa a tareas asignadas a distintos colaboradores sin el compromiso mensual, casi un consolidado de trabajadores part-time. 

Algo bueno de este boom de empleo es que da segundas oportunidades a ex militares, ex sentenciados, y primeras oportunidades sin discriminación a personas con alguna discapacidad. 
Luego habla de la necesidad de legislar al respecto y de tomar consciencia por parte de los economistas que sub-estimaron al mercado. (Los economistas pensaban (si es que piensan)que el mercado del trabajo se mantendría bajo)
\subsection{Wage Stagnation Is One Disease With Many Causes} Se habla de estancamiento de los salarios cuando las remuneraciones por las tareas se mantienen o bajan.
se habla que los culpables pueden ser, automatización, la competencia extrajera. 
\smallskip
Luego muestra un gráfico un poco lineal, que trata de como históricamente los salarios han subido de a 2,5 USD hora. No considera inflación, tampoco leyes sociales, impuestos, etc. La solución a esto es que nos muestran los precios encadenados y obvio sube, pero como que se estanca. Claro por eso el titulo. 
\smallskip
Tal como se los conté arriba los costos de la salud, hacen que uno deba desembolsar mas plata para tener salud. por ende aunque me pagan mas, recibo menos. Bueno obviamente eso aumenta el costo de la mano de obra.  Finalmente habla que esos 3 puntos nunca se han recuperado del todo y han generado el estancamiento. 